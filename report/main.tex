\documentclass{article}
\usepackage[right=2cm, left=2cm, top=2cm, bottom=2cm]{geometry}
\usepackage{pgfplots}
%\pgfplotsset{width=7cm,compat=1.17}\usepgfplotslibrary{statistics}
\usepgfplotslibrary{units}
\usepgfplotslibrary{groupplots}

%%%%%%%%%%%%%%%%%%%%%%%%%%%%%%%%%%%%%%%%%%%
% The following allows combinations of single
% columns, multicolumn and figures.
\usepackage{caption}
\usepackage{multicol}
\newenvironment{Figure}
  {\par\medskip\noindent\minipage{\linewidth}}
  {\endminipage\par\medskip}
%%%%%%%%%%%%%%%%%%%%%%%%%%%%%%%%%%%%%%%%%%%

\title{Event driven simulation of a granular gas}
\author{Jonas Bueie}
\date{\today}

\begin{document}
\large
\maketitle

\noindent \hrulefill
\section*{Abstract}

This report summarizes a computational experiment where a two dimensional granular gas has been simulted in an event driven simulation.

\noindent \hrulefill

\begin{multicols}{2}
\section{Introduction}

A granular gas is a two dimensional model in which a gas is represented by circular particles of finite radii and masses, that can collide elastically or inellastically.
In this report, some results from simulations of suh a gas are given.
The simulations show results that are in agreement with staatistical mechanics.

\section{Theory}

\section{Results}

%% Task 1
\begin{tikzpicture}
    \begin{axis}[
        xlabel={Speed},
        ylabel={Number of particles},
        x unit = m/s,
        ybar interval,
        xtick=,
        xtick distance = 1,
        xmin=0,
        xmax=6,
        ymin=0,
        grid=minor,
    ]
    \addplot+[
        hist={bins=40},
        xtick=\empty,
    ]
        table[
            y =v_0,
        ]{../data/task_1_speeds.csv};
    \end{axis}
\end{tikzpicture}

\begin{tikzpicture}
    \begin{axis}[
        xlabel={Speed},
        ylabel={Number of particles},
        x unit = m/s,
        ybar interval,
        hist/intervals=true,
        ymin=0,
        xtick=,
        grid=minor,
    ]
    \addplot+[hist={bins=12}]
        table[
            y =v_1,
        ]{../data/task_1_speeds.csv};
    \end{axis}

    \begin{axis}[
        xtick=\empty,
        ytick=\empty,
    ]
    \addplot[
        domain=0:10,
        samples=100,
        color=black,
        xtick=\empty,
        ]
        {(2*x/0.1)*exp(-((x)^2)/10)};
        % TODO: Calculate <v> and adjust this ad hoc scaling.
    \end{axis}
\end{tikzpicture}

%% Task 2
\begin{tikzpicture}
    \begin{axis}[
        xlabel={Speed},
        ylabel={Number of particles},
        x unit = m/s,
        ybar interval,
        hist/intervals=true,
        ymin=0,
        xtick=,
        grid=minor,
    ]
    \addplot+[hist={bins=12}]
        table[
            y = v_1,
            xtick=\empty,
        ]{../data/task_2_speeds.csv};
    \end{axis}
\end{tikzpicture}


\begin{tikzpicture}
    \begin{groupplot}[
        group style={
            group name = task_2,
            group size=1 by 2,
            vertical sep=0pt,
            xlabels at=edge bottom,
        },
        height=6cm,
        width=\linewidth,
        ybar interval,
        separate axis lines,
        xtick=,
        xmin=0,
        xmax=4,
        ymin=0,
        ymax=40,
        ylabel={Number of particles},
        xlabel={Speed},
    ]


    \nextgroupplot
        \addplot+[hist={bins=16}]
            table[
                y = v_1,
            ]{../data/task_2_speeds.csv};
    \nextgroupplot
        \addplot+[hist={bins=16}]
            table[
                y = v_2,
            ]{../data/task_2_speeds.csv};
    \end{groupplot}

    \begin{axis}[
        xlabel={Speed},
        ylabel={Number of particles},
        x unit = m/s,
        ybar interval,
        hist/intervals=true,
        grid=minor,
        anchor=north west,
        hide y axis,
        hide x axis,
    ]
%    \addplot[
%        domain=0:10,
%        samples=100,
%        color=black,
%        xtick=\empty,
%        ]
%        {(2*x/0.1)*exp(-((x)^2)/10)};
    \end{axis}

\end{tikzpicture}


%% Task 3
\begin{tikzpicture}
\begin{axis}[
    xlabel={Time},
    ylabel={Kinetic energy},
]
\addplot [
    mark=none,
    ]table{../data/task_30.8_energy.csv};
\addlegendentry{Maxwell distribution}
\end{axis}
\end{tikzpicture}

\end{multicols}
\end{document}

