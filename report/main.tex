\documentclass{article}
\usepackage[right=2cm, left=2cm, top=2cm, bottom=2cm]{geometry}
\usepackage{pgfplots}
%\pgfplotsset{width=7cm,compat=1.17}\usepgfplotslibrary{statistics}
\usepgfplotslibrary{units}
\usepgfplotslibrary{groupplots}
\pgfplotsset{compat=1.3}
%\pgfplotsset{every axis/.append style={line width=1pt}}
%\pgfplotsset{every tick/.append style={thin}}

%%%%%%%%%%%%%%%%%%%%%%%%%%%%%%%%%%%%%%%%%%%
% The following allows combinations of single
% columns, multicolumn and figures.
\usepackage{caption}
\usepackage{multicol}
\newenvironment{Figure}
  {\par\medskip\noindent\minipage{\linewidth}}
  {\endminipage\par\medskip}
%%%%%%%%%%%%%%%%%%%%%%%%%%%%%%%%%%%%%%%%%%%

\title{Event driven simulation of a granular gas}
\author{Jonas Bueie}
\date{\today}

\begin{document}
\large
\maketitle

\noindent \hrulefill
\section*{Abstract}

This report summarizes a computational experiment where a two dimensional granular gas has been simulted in an event driven simulation.

\noindent \hrulefill

\begin{multicols}{2}
\section{Introduction}

A granular gas is a two dimensional model in which a gas is represented by circular particles of finite radii and masses, that can collide elastically or inellastically.
In this report, some results from simulations of suh a gas are given.
The simulations show results that are in agreement with staatistical mechanics.

\section{Theory}

\section{Results}
%% Include the separate tasks, since the figures make the file long.
%% Task 1
\begin{frame}
\begin{figure}
    \centering
    \begin{tikzpicture}
        \begin{axis}[
            title = Initial speed distribution,
            xlabel={Speed},
            ylabel={Number of particles},
            x unit = m/s,
            ybar interval,
            xtick=,
            xtick distance = 1,
            xmin=0,
            xmax=6,
            ymin=0,
            grid=minor,
        ]
        \addplot+[
            hist={bins=40},
            xtick=\empty,
        ]
            table[
                y =v_0,
            ]{../data/task_1_speeds.csv};
        \end{axis}
    \end{tikzpicture}
    %\captionof{figure}{The initial speed distribution of the particles is a Dirac's delta function $\delta(v-1)$.}
    %\label{task_1:init_vel}
\end{figure}
\end{frame}

\begin{frame}
\begin{figure}
    \centering
    \begin{tikzpicture}
        \begin{axis}[
            title = Speed distribution at equilibrium,
            xlabel={Speed},
            ylabel={Number of particles},
            x unit = m/s,
            ybar interval,
            hist/intervals=true,
            ymin=0,
            xtick=,
            grid=minor,
        ]
        \addplot+[hist={bins=12}]
            table[
                y =v_1,
            ]{../data/task_1_speeds.csv};
        \end{axis}

        \begin{axis}[
            xtick=\empty,
            ytick=\empty,
        ]
        \addplot[
            domain=0:10,
            samples=100,
            color=black,
            xtick=\empty,
            ]
            {(2*x/0.1)*exp(-((x)^2)/10)};
        \addlegendentry{Maxwell distribution}
            % TODO: Calculate <v> and adjust this ad hoc scaling.
        \end{axis}
    \end{tikzpicture}
    %\captionof{figure}{At equilibrium with $\xi = 1.0$, the speed distribution of the particles is a Maxwell distribution.}
    %\label{task_1:final_vel}
\end{figure}
\end{frame}


%% Task 2

\begin{Figure}
    \centering
    \begin{tikzpicture}
        \begin{groupplot}[
            group style={
                group name = task_2,
                group size=1 by 2,
                vertical sep=0pt,
                xlabels at=edge bottom,
            },
            height=6cm,
            width=\linewidth,
            ybar interval,
            separate axis lines,
            xtick=,
            xmin=0,
            xmax=4,
            ymin=0,
            ymax=40,
            xlabel={Speed},
        ]
        \nextgroupplot
            \addplot+[hist={bins=16}]
                table[
                    y = v_1,
                ]{../data/task_2_speeds.csv};
        \nextgroupplot[
            ylabel=Number of particles,
            every axis y label/.append style={at=(ticklabel cs:1.0)}
        ]
            \addplot+[hist={bins=16}]
                table[
                    y = v_2,
                ]{../data/task_2_speeds.csv};
        \end{groupplot}
        \begin{axis}[
            xlabel={Speed},
            ylabel={Number of particles},
            x unit = m/s,
            ybar interval,
            hist/intervals=true,
            grid=minor,
            anchor=north west,
            hide y axis,
            hide x axis,
        ]
    %    \addplot[
    %        domain=0:10,
    %        samples=100,
    %        color=black,
    %        xtick=\empty,
    %        ]
    %        {(2*x/0.1)*exp(-((x)^2)/10)};
        \end{axis}
    \end{tikzpicture}
    \captionof{figure}{Histogram showing the speed distribution of a system containing equalt parts of two types of particles:
        One with mass $m=m_0$ (above), and one with mass $m=4m_0$ (below).}
    \label{fig:task_2:vel_distribution}
\end{Figure}

%% Task 3
\begin{Figure}
    \centering
    \begin{tikzpicture}
        \begin{axis}[
            xlabel={Time},
            ylabel={Energy},
            %y unit = J, %% TODO: What unit is this really?
            ymin=0,
            legend style={
                at={(0.99,0.01)},
                anchor=south east,
            },
        ]
        \addplot [
            mark=none,
            style=dashed,
            line width = 1pt,
            ]table[
                y = e_tot,
            ]{../data/task_3_1_energy.csv};
        \addlegendentry{Total kinetic energy}
        \addplot [
            mark=none,
            color=red,
            line width = 1pt,
            ]table[
                y = e_1,
            ]{../data/task_3_1_energy.csv};
        \addlegendentry{$m=4m_0$}
        \addplot [
            mark=none,
            color=blue,
            line width = 1pt,
            ]table[
                y = e_0,
            ]{../data/task_3_1_energy.csv};
        \addlegendentry{$m=m_0$}
        \end{axis}
    \end{tikzpicture}
    \captionof{figure}{The figure shows the energy of the two subsystems (of different masses), as a function of time, with $\xi = 1.0$.}
    \label{fig:task_3:1}
\end{Figure}


\begin{Figure}
    \centering
    \begin{tikzpicture}
        \begin{axis}[
            xlabel={Time},
            ylabel={Energy},
            %y unit = J, %% TODO: What unit is this really?
            ymin=0,
            legend style={
                at={(0.99,0.99)},
                anchor=north east,
            },
        ]
        \addplot [
            mark=none,
            style=dashed,
            line width = 1pt,
            ]table[
                y = e_tot,
            ]{../data/task_3_0.9_energy.csv};
        \addlegendentry{Total kinetic energy}
        \addplot [
            mark=none,
            color=red,
            line width = 1pt,
            ]table[
                y = e_1,
            ]{../data/task_3_0.9_energy.csv};
        \addlegendentry{$m=4m_0$}
        \addplot [
            mark=none,
            color=blue,
            line width = 1pt,
            ]table[
                y = e_0,
            ]{../data/task_3_0.9_energy.csv};
        \addlegendentry{$m=m_0$}
        \end{axis}
    \end{tikzpicture}
    \captionof{figure}{The figure shows the energy of the two subsystems (of different masses), as a function of time, with $\xi = 0.9$.}
    \label{fig:task_3:0.9}
\end{Figure}

\begin{Figure}
    \centering
    \begin{tikzpicture}
        \begin{axis}[
            xlabel={Time},
            ylabel={Energy},
            %y unit = J, %% TODO: What unit is this really?
            ymin=0,
            legend style={
                at={(0.99,0.99)},
                anchor=north east,
            },
        ]
        \addplot [
            mark=none,
            style=dashed,
            line width = 1pt,
            ]table[
                y = e_tot,
            ]{../data/task_3_0.8_energy.csv};
        \addlegendentry{Total kinetic energy}
        \addplot [
            mark=none,
            color=red,
            line width = 1pt,
            ]table[
                y = e_1,
            ]{../data/task_3_0.8_energy.csv};
        \addlegendentry{$m=4m_0$}
        \addplot [
            mark=none,
            color=blue,
            line width = 1pt,
            ]table[
                y = e_0,
            ]{../data/task_3_0.8_energy.csv};
        \addlegendentry{$m=m_0$}
        \end{axis}
    \end{tikzpicture}
    \captionof{figure}{The figure shows the energy of the two subsystems (of different masses), as a function of time, with $\xi = 0.8$.}
    \label{fig:task_3:0.8}
\end{Figure}


\begin{tikzpicture}
    \begin{axis}[
        xlabel={x},
        ylabel={y},
        xtick=,
        xmin=0,
        xmax=1,
        ymin=0,
        ymax=1,
        grid=minor,
    ]
    \addplot+[
        scatter,
        only marks,
        scatter src=explicit,
        scatter/use mapped color={
            draw=black,
            fill=blue,
        },
        visualization depends on=\thisrow{radius}\as\radius,
        scatter/@pre marker code/.append style={
            /tikz/mark size=\radius*170
        }
    ]
        table[
            meta=count,
        ]{../data/task_4_particles.csv};
    \end{axis}
\end{tikzpicture}


\begin{tikzpicture}
    \begin{axis}[
        xlabel={x},
        ylabel={y},
        xtick=,
        xmin=0,
        xmax=1,
        ymin=0,
        ymax=1,
        grid=minor,
        colorbar,
    ]
    \addplot+[
        scatter,
        only marks,
        scatter src=explicit,
        scatter/use mapped color={
            draw=black,
            fill=mapped color,
        },
        visualization depends on=\thisrow{radius}\as\radius,
        scatter/@pre marker code/.append style={
            /tikz/mark size=\radius*170
        }
    ]
        table[
            meta=count,
        ]{../data/task_4_particles.csv};
    \end{axis}
\end{tikzpicture}


\end{multicols}
\end{document}

